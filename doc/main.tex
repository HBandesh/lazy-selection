\documentclass[12pt,a4paper]{article}

\usepackage[T1]{fontenc}
\usepackage[utf8]{inputenc}
\usepackage[margin=2.5cm]{geometry}
\usepackage{graphicx}
\usepackage[hidelinks]{hyperref}
\usepackage{fancyhdr}
\usepackage{lastpage}
\usepackage{appendix}
\usepackage{color}
\usepackage{palatino}
\usepackage{changepage}
\usepackage{subcaption}
\usepackage{enumitem}
\usepackage{csquotes}
\usepackage{verbatim}
\usepackage[cache=false]{minted}
\usepackage[ruled,vlined]{algorithm2e}

\usemintedstyle{default}
\setminted{
  linenos=true,
  breaklines=true,
  fontsize=\small,
  frame=single,
}

\title{Lazy Selection}
\author{Carlos Requena López}

%% Fancy layout
\pagestyle{fancy}
\lhead{Lazy selection}
\chead{}
\rhead{}
\lfoot{}
\cfoot{}
\rfoot{Page \thepage\ of \pageref{LastPage}}
\renewcommand{\headrulewidth}{0.4pt}
\renewcommand{\footrulewidth}{0.4pt}


%%% --- %%% --- DOCUMENT START --- %%% --- %%%
\begin{document}
\maketitle
\thispagestyle{empty}
\tableofcontents
\newpage
%%% Counting pages now %%%
\pagestyle{fancy}
\setcounter{page}{1}


\section{Introduction}

\section{Analysis and expectations}

% \ref{algo}

\begin{algorithm}[h]
  \SetAlgoLined
  \KwIn{A set S of segments}
  \KwOut{The autopartition binary tree}

  \nl Pick a random permutation of S and take the first element\;
  \nl Extend this element into a line and create a node with it in the tree\;
  \nl Filter the segments strictly to the left in the \texttt{left}
  child\;
  \nl Filter the segments strictly to the right in the \texttt{right}
  child\;
  \nl \For{segment in all segments that intersect}{
    \nl add its subsegment laying on the left part to the
    \texttt{left} child\;
    \nl add its subsegment laying on the right part to the
    \texttt{right} child\;
  }
  \nl \While{\texttt{left} or \texttt{right} have segments in them}{
    \nl Repeat this procedure recursively on both sets\;
  }
\caption{\bf LazySelect}
\label{algo}
\end{algorithm}

% \ref{eq:upper-bound}.

\begin{equation}
  \label{eq:upper-bound}
  n + 2 \sum_u \sum_{i=1}^{n-1} \frac{1}{i+1} \leq n + 2nH_n
\end{equation}

% \cite[p.~50]{Motwani:1995:RA:211390}

\section[Implementation]{Implementation}

% \footnote{Code can be found at
%   \url{https://github.com/carlosgeos/bspace-part}}\footnote{An
%   executable \texttt{.jar} file can be downloaded from
%   \url{https://github.com/carlosgeos/bspace-part/releases}}

% \ref{fig:sample-input}

\begin{minted}[label=sample-output]{clojure}
"BSpace partitions"
{:seg {:a {:x 492, :y 556}, :b {:x 388, :y 932}},
 :left
 {:seg {:a {:x 625, :y 409}, :b {:x 530, :y 745}},
  :left (),
  :right
  {:seg {:a {:x 626, :y 355}, :b {:x 588, :y 243}},
   :left (),
   :right ()}},
 :right
 {:seg {:a {:x 264, :y 626}, :b {:x -8, :y 571}},
  :left (),
  :right
  {:seg {:a {:x 88, :y 291}, :b {:x 36, :y 79}},
   :left (),
   :right ()}}}
\end{minted}

% \begin{figure}[ht!]
%   \centering
%   \includegraphics[width=0.7\textwidth]{img/sample-input.png}
%   \caption{Five random segments on a 720x720 canvas}
%   \label{fig:sample-input}
% \end{figure}

\section{Results}

% \ref{tab:results}

% \begin{table}[ht!]
%   \centering
%   \begin{tabular}[center]{||c|c|c|c|c||}
%     \hline
%     \# of segments & Canvas size (px) & Max segment size (px) & Tree size
%     & Upper bound \\
%     \hline \hline
%     10 & 720x720 & 50 & 12 & 68 \\
%     10 & 720x720 & 50 & 11 & 68 \\
%     10 & 720x720 & 250 & 14 & 68 \\
%     100 & 720x720 & 50 & 132 & 1137 \\
%     100 & 720x720 & 50 & 137 & 1137 \\
%     100 & 720x720 & 250 & 221 & 1137 \\
%     1000 & 720x720 & 50 & 2213 & 15970 \\
%     1000 & 720x720 & 50 & 2262 & 15970 \\
%     1000 & 720x720 & 250 & 2808 & 15970 \\
%     2000 & 720x720 & 50 & 5179 & 34713 \\
%     2000 & 720x720 & 5 & 2395 & 34713 \\
%     2000 & 720x720 & 250 & 6710 & 34713 \\
%     5000 & 720x720 & 250 & 16227 & 95945 \\
%     10000 & 720x720 & 3 & 12698 & 205752 \\
%     \hline
%   \end{tabular}
%   \caption{Simulation results}
%   \label{tab:results}
% \end{table}

\section{Conclusion}

\appendix
\section{Appendix - code listing}

% \inputminted[label=common.clj]{clojure}{../src/bspace_part/common.clj}
% \inputminted[label=geometry.clj]{clojure}{../src/bspace_part/geometry.clj}
% \inputminted[label=core.clj]{clojure}{../src/bspace_part/core.clj}

\nocite{*}
\bibliographystyle{plain}
\bibliography{refs}

\end{document}
